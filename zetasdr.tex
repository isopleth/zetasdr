\documentclass[11pt, twoside]{article}

\usepackage[parfill]{parskip}
\usepackage{amsmath}
\usepackage{gensymb}
\usepackage{geometry} 
\usepackage{graphicx, caption}
\usepackage{listings}
\usepackage{natbib}
\usepackage{textgreek}

\lstset{
  basicstyle=\small\ttfamily,
  columns=flexible,
  breaklines=true
}

\geometry{
  a4paper,
  total={170mm,257mm},
  left=20mm,
  top=20mm,
}

\begin{document}

\title{Simulating the ZetaSDR radio}
\author{Jason Leake}
\date{March 2019}
\maketitle
\section{Introduction}
The ZetaSDR radio \citep{ly1gp:2007}, designed by a Lithuanian radio
amateur, call sign LY1GP, is a simple direct conversion radio receiver
with a very small part count and forms the front end for a software
defined radio. The output from it is an analogue inphase and
quadrature signal, which allows the baseband to be extracted from
several different forms of modulation, for example frequency
modulation and quadrature amplitude modulation.

This program is simulates the operation of the 74HC4052
\citep{Motorola:1996}, and the Johnson counter that drives it,
producing the response to a simulated amplitude modulated RF
signal. The source code for the simulation is {\texttt program.cpp},
and {\texttt plot.py} generates the plots from its CSV format output
files.  The program works on instantaneous samples of a simulated
incoming signal using the transformations that key components apply to
it, in contrast to circuit solvers such as Spice.

The program was written for Ubuntu Linux is dependent upon several
packages, which can be installed via:

\begin{lstlisting}
sudo apt-get install make g++ librtlfilter-dev python3 python3-matplotlib texlive-all
\end{lstlisting}

Compilation and processing up to and including producing the graphs is
done using the command:

\begin{lstlisting}
make
\end{lstlisting}

Generating this PDF file as well, is done using:

\begin{lstlisting}
make zetasdr.pdf
\end{lstlisting}

\section{Operation of the ZetaSDR}

The incoming RF electromagnetic signal induces a small electrical
current in the antenna.  The current passes across capacitor
C6,\footnote{circuit diagram refers to the schematic for the ZetaSDR
  at \cite{ly1gp:2007}.} which acts as a high pass filter, blocking DC
from passing from the receiver circuit to the antenna, which could
otherwise cause problems with preamplifiers.

A 2.5 volt bias is applied to the RF signal by the voltage divider
R1/R2 so that it oscillates around 2.5 volts instead of 0 volts.  This
means that the rest of the circuit is dealing with a signal in the
middle of its normal operating range instead of around 0 volts where
non-linearity in the response will be greatest even if it can respond
at all.

\subsection{Tayloe quadrature product detector}

The ZetaSDR uses a Tayloe quadrature product detector
\citep{Tayloe:2001, Tayloe:2013}, often informally called a Tayloe
mixer, to demodulate the signal.  This is a type of quadrature
sampling detector, a switching mixer that produces quadrature (IQ)
signals. It is based on a 74HC4052 analogue multiplexer/demultiplexer
and capacitors to implement an integrator.  It samples, averages and
holds the RF signal on each quarter cycle, sending the first and third
quarter samples to the I output, and the second and fourth quarter
samples to the Q output, inverting the samples read on the third and
fourth cycles.

The 74HC4052 is a two channel analogue multiplexer/demultiplexer.  One
channel has an input, X, and switches it to one of four outputs, X0
$\rightarrow$ X3 depending on the state of the digital inputs A and B
(as shown in Table~\ref{table:74HC4052}).  If output X{\it n} is
enabled then the impedance between it and X is around 70 {\ohm}
(depending on supply voltage), and if not enabled then X{\it n} has an
extremely high impedance.  The device is bidirectional, but this is
not relevant here, for the purposes of the ZetaSDR it is enough that
the input from the antenna (X) is switched to one of the four outputs
X1 $\rightarrow$ X4. The other channel has an input called Y,
switching to Y1 $\rightarrow$ Y4.  The circuit uses both channels in
parallel to reduce the impedance seen across the device from about 70
{\ohm} to 35 {\ohm}.

\begin{table}[ht]
  \center
  \begin{tabular}{|l| l| l| l|}
    \hline
    A & B & X & Y  \\
    \hline
    \hline
    0 &0& X0& Y0\\
    0 &1& X1& Y1\\
    1 &0& X2& Y2\\
    1 &1& X3& Y3\\
    \hline
  \end{tabular}
  \caption{Operation of 74HC4052}
  \label{table:74HC4052}
\end{table}

The inputs A and B are driven by a Johnson counter (also called a
twisted ring counter) made from two D-type flip flops.  This provides
two binary signals, connected to A and B, that change on every clock
cycle.  The clock is supplied by the QG2 local oscillator. The counter
operates as a 2 bit wide shift register, with the the last bit value
being inverted and fed into the first bit of the shift register on
each clock cycle, so generating a repeating sequence 00, 01, 11, 10.
The local oscillator runs at four times the frequency of the RF
carrier, producing a new bit pattern in the Johnson counter, and thus
switching the 74HC4052, on every quarter cycle of the RF carrier.

Hence, each of the four pairs of outputs X0/Y0, X1/Y1, X2/Y2 and X3/Y3
receive a quarter cycle of the RF signal.  The outputs connect to
capacitors that charge/discharge as the RF carrier signal is applied
to them but hold their voltage when the corresponding 74HC4052 output
is disabled and so in its high impedance state.  The amplitude of the
RF carrier depends on the incoming signal strength and the quality of
the antenna and preamplifier, so I picked an arbitrary value of 1 mV
for the simulation program.

Since the local oscillator output is the Johnson counter clock, the
flip flops making up the counter change when the local oscillator
signal transitions from a voltage corresponding to logic 0 to one
representing logic 1.  The simulation sets this at a conventional 2.4
volts, and since the local oscillator has a swing of 5 volts, the
clock occurs just below the midpoint on the local oscillator upswing.

Because the Johnson counter produces the sequence 00, 01, 11, 10, the
first and third quarters of the RF carrier cycle are selected using
outputs X0 and X3 respectively, and the second and fourth using X1 and
X2.  If a binary counter that produced the sequence 00, 01, 10, 11,
was used instead then the outputs would be X0/X2 and X1/X3.

The voltage on capacitors C2 and C3 are shown in
Figure~\ref{figure:C2C3simple}, in a somewhat idealised simulation of
the circuit.  The local oscillator is set to have a range of 0--5
volts, and have its 0{\degree} point at the 0{\degree}, 90{\degree},
180{\degree} and 270{\degree} positions of the RF carrier.  In this
idealised simulation the Johnson counter therefore changes state very
slightly before this point, when the local oscillator voltage is
rising through 2.4 volts.

In practice, it is unlikely that the local oscillator will be in phase
with the RF carrier, and there will be small delays in the Johnson
counter logic and the 74HC4052 operation.

The two capacitors are also connected to the differential inputs of an
active low pass filter that both amplifies and filters the difference
between the two voltages.  This voltage is shown on the lower plot in
Figure~\ref{figure:C2C3simple} -- the effect of taking the difference
between the two signals, which will be normally be similar but of
opposite polarity, is to produce a signal of double the size of each.
The corresponding voltages on the other pair of capacitors, C4 and C5,
are shown in Figure~\ref{figure:C4C5simple}.

\begin{figure}
  \center
  \captionsetup{width=.8\linewidth}
  \includegraphics[width=.8\linewidth]{c2c3Voltage.png}
  \caption{Voltage on C2 and C3, ZetaSDR, when the local oscillator is
    in phase with the RF signal. The 74HC4052 switching leads the RF
    signal phase very slightly because its local oscillator derived
    clock changes logic state just below the midpoint on the local
    oscillator upswing.  }
  \label{figure:C2C3simple}
\end{figure}


\begin{figure}
  \center
  \captionsetup{width=.8\linewidth}
  \includegraphics[width=.8\linewidth]{c4c5Voltage.png}
  \caption{Voltage on C4 and C5, ZetaSDR, when the local oscillator is
    in phase with the RF signal}
  \label{figure:C4C5simple}
\end{figure}

The sharp transitions and the short-lived spikes will be attenuated by
the active low pass filter stage.  Hence, the demodulator is producing
an output where the I signal is sampled at two points in the carrier
cycle 180{\degree} apart, and the Q signal is the signal sampled
90{\degree} on from those points in much the same way that the
multiplication by the respective local oscillator does it in a
conventional multiplying IQ mixer.

Hence, if the amplitude or phase of the signal does not change
rapidly during the sampling and we ignore the quarter cycles where the
voltage on the capacitors is changing, then for the I signal, the
voltage at capacitor C2  in Figure~\ref{figure:C2C3simple} is

\begin{equation*}
  s_1 = A cos(\theta) sin({\omega_b}t)
\end{equation*}

where:\\
\\
  ${\omega_b}$ is the baseband frequency in radians per second\\
$\theta$ is the phase difference between the carrier and the Johnson counter transitions (and hence the 74HC4052 switching)

and at C3 is:

\begin{align*}
  s_2 & = A cos(\theta) sin(\pi + {\omega_b}t) \\
  \\
  &= -A cos(\theta) sin({\omega_b}t)
\end{align*}

Because the active filter is the difference between the two voltages,

\begin{equation}\label{eqn:tayloei}
  I = s_1 - s_2 = 2 A cos(\theta) sin({\omega_b}t)
\end{equation}

Hence, the difference into the active low pass filter, ignoring the
high frequency portion which results when one of the capacitors is
tracking the output from the 74HC4052 rather than holding its voltage,
is $2A sin(\theta) sin({\omega_b}t)$.

By the same reasoning, the Q signal formed from the voltage difference
between C4 and C5 is:

\begin{equation}\label{eqn:tayloeq}
  Q = 2 A sin(\theta) sin({\omega_b}t)
\end{equation}

\begin{figure}
  \center
  \captionsetup{width=.8\linewidth}
  \includegraphics[width=.8\linewidth]{c2c3ModVoltage.png}
  \caption{Voltage on C2 and C2 in the ZetaSDR, when the local
    oscillator is in phase with the RF carrier, and the 7 MHz carrier
    is amplitude modulated at 200 kHz}
  \label{figure:C2C3mod}
\end{figure}

\begin{figure}
  \center
  \captionsetup{width=.8\linewidth}
  \includegraphics[width=.8\linewidth]{c4c5ModVoltage.png}
  \caption{Voltage on C4 and C5, when the local oscillator is in phase
    with the RF carrier, and the 7 MHz carrier is amplitude modulated
    at 200 kHz}
  \label{figure:C4C5mod}
\end{figure}


The corresponding plots for a modulated signal are shown in
Figures~\ref{figure:C2C3mod} and \ref{figure:C4C5mod}. Many more RF
cycles are shown in these plots, and the 7 MHz carrier is amplitude
modulated at 200 kHz which is an order of magnitude above any an audio
signal, but has been done so that several AM cycles can be accommodated


\begin{figure}
  \center
  \captionsetup{width=.8\linewidth}
  \includegraphics[width=.8\linewidth]{c2c3VoltagePhase.png}
  \caption{Voltage on C2 and C3 in the ZetaSDR, when the local
    oscillator leads the RF carrier by 35{\degree}}
  \label{figure:C2C3phase}
\end{figure}

\begin{figure}
  \center
  \captionsetup{width=.8\linewidth}
  \includegraphics[width=.8\linewidth]{c4c5VoltagePhase.png}
  \caption{Voltage on C4 and C5 in the ZetaSDR, when the local
    oscillator leads the RF carrier by 35{\degree}}
  \label{figure:C4C5phase}
\end{figure}

In practice, the 74HC4052 switching will usually be more out of phase
with the RF carrier signal, and so something such as
Figure~\ref{figure:C2C3phase} (and Figure~\ref{figure:C4C5phase} for
the other pair of capacitors) is more likely.  The modulated versions
of these plot, over a larger number of carrier cycles, are shown in
Figures~\ref{figure:C2C3modphase} and \ref{figure:C4C5modphase} are
obtained.

Finally, Figure~\ref{figure:tayloelowpassphase} shows the I and Q
signals through a low pass filter and then combining them to extract
the baseband.  Because of the unusually high modulation frequency and
signal strength used in the simulation, a 400 kHz $2^{nd}$ order zero
gain low pass Butterworth filter is used instead of the 10 kHz cutoff
active $1^{st}$ order low pass filter in the ZetaSDR radio.

\begin{figure}
  \center
  \captionsetup{width=.8\linewidth}
  \includegraphics[width=.8\linewidth]{c2c3ModVoltagePhase.png}
  \caption{Voltage on C2 and C3 in the ZetaSDR, when the local
    oscillator leads the carrier by 35{\degree}, and the 7 MHz carrier
    is amplitude modulated at 200 kHz}
  \label{figure:C2C3modphase}
\end{figure}

\begin{figure}
  \center
  \captionsetup{width=.8\linewidth}
  \includegraphics[width=.8\linewidth]{c4c5ModVoltagePhase.png}
  \caption{Voltage on C4 and C5 in the ZetaSDR, when the local
    oscillator leads the carrier by 35{\degree}, and the 7 MHz carrier
    is amplitude modulated at 200 kHz}
  \label{figure:C4C5modphase}
\end{figure}


\begin{figure}
  \center
    \captionsetup{width=.8\linewidth}
  \includegraphics[width=.8\linewidth]{tayloeLowPassPhase.png}
  \caption{Response to a 200 kHz amplitude modulated signal, when the
    local oscillator is 35{\degree} ahead of the RF carrier. Because
    of the high input signal amplitude and the high modulation
    frequency, the active low pass filters have been replaced with a
    $2^{nd}$ order low pass filter, but with unity gain and a cutoff
    frequency of 400 kHz.}
  \label{figure:tayloelowpassphase}
\end{figure}

\section{Comparison with an ideal multiplying IQ mixer}

A sinusoidally amplitude modulated RF signal, with a 100\% modulation
depth, double-sideband and full-carrier, is described by

\begin{equation*}
S = A sin({\omega_b}t) sin({\omega_c}t)
\end{equation*}

where:\\
\\
$A$ is the signal amplitude\\
$\omega_b$ is the baseband frequency\\
$\omega_c$ is the carrier frequency \\
  $t$ is time\\

In a conventional multiplying IQ mixer, the I signal is produced by
mixing with a local oscillator with a phase difference $\theta$:

\begin{align*}
I =& A sin({\omega_b}t)  sin({\omega}_ct)sin({\omega}_ct + \theta) \\
\\
&= A sin({\omega_b}t)  \frac{cos(\theta) - cos(2 {\omega}_ct - \theta)}{2}
\\
&= \frac{A sin({\omega_b}t) cos(\theta)}{2} - \frac{cos(2 {\omega_c}t - \theta)}{2}
\end{align*}

Because

\begin{equation*}
  sin(\alpha) sin(\beta) = \frac{sin(\alpha + \beta) - sin(\alpha - \beta)}{2}
\end{equation*}

The Q signal is produced by mixing the incoming RF signal with a
similar local oscillator signal that is 90{\degree} out of phase with
the I local oscillator:


\begin{align*}
  Q& = A sin({\omega_b}t) cos({\omega}_ct) sin({\omega}_ct + \theta) \\
  \\
  & = A sin({\omega_b}t) \frac{sin(2{\omega}_ct + \theta) + sin(\theta)}{2}\\
  \\
& = \frac{A sin({\omega_b}t)sin(2{\omega}_ct + \theta)}{2} + \frac{A sin({\omega_b}t)sin(\theta)}{2}\\
\end{align*}

Because

\begin{equation*}
  cos(\alpha)sin(\beta) = \frac{(sin(\alpha + \beta) - sin(\alpha - \beta)}{2} 
\end{equation*}

In both cases, the low pass filter will remove the portion of the
signal with frequency $2{\omega_c}$ leaving

\begin{equation*}
  I = \frac{A sin({\omega_b}t)cos(\theta)}{2}
\end{equation*}

\begin{equation*}
  Q = \frac{A sin({\omega_b}t)sin(\theta)}{2}
\end{equation*}

These two equations are of the same form as
Equations~\ref{eqn:tayloei} and \ref{eqn:tayloeq}, the ones for the
Tayloe quadrature product detector.

Like the IQ mixer output, the high frequency portion of the Tayloe
detector signal is at $2\omega_c$, since the rapidly varying portion gets
through two cycles for each cycle of the RF carrier.  although in this
case there are additional harmonics.

The results of the simulation of an ideal multiplying IQ mixer are
shown in Figures~\ref{figure:iqmod}--\ref{figure:iqmodlowpassphase}
for comparison with the Tayloe detector simulation plots above.

\section{Conclusion}

The Tayloe quadrature product detector produces a similar waveform to
a conventional multiplying IQ mixer.

\begin{figure}
  \center \captionsetup{width=.8\linewidth}
  \includegraphics[width=.8\linewidth]{iqModVoltage.png}
  \caption{I/Q outputs using an ideal multiplying IQ mixer, when the
    local oscillator is in phase with the RF carrier, and the 7 MHz
    carrier is amplitude modulated at 200 kHz}
  \label{figure:iqmod}
\end{figure}


\begin{figure}
  \center
    \captionsetup{width=.8\linewidth}
  \includegraphics[width=.8\linewidth]{iqModVoltagePhase.png}
  \caption{I/Q outputs using an ideal multiplying IQ mixer when the
    local oscillator leads the carrier by 35{\degree}, and the 7 MHz
    carrier is amplitude modulated at 200 kHz}
  \label{figure:iqmodphase}
\end{figure}


\begin{figure}
  \center
    \captionsetup{width=.8\linewidth}
  \includegraphics[width=.8\linewidth]{iqModLowPassPhase.png}
  \caption{I/Q outputs using an ideal multiplying IQ mixer, when the
    local oscillator is 35{\degree} ahead of the RF carrier, and with
    the I/Q signals passed through a $4^{th}$ order low pass filter
    with a cutoff of 3.5 MHz to remove the $2{\omega_c}$ signal.
    Compare this ideal case with
    Figure~\ref{figure:tayloelowpassphase}.}
  \label{figure:iqmodlowpassphase}
\end{figure}

\begin{thebibliography}{9}
\bibitem[LY1GP(2007)]{ly1gp:2007}
  LY1GP (2007), {\it ZetaSDR for 40m band}
  \\\texttt{http://www.qrz.lt/ly1gp/SDR/}
\bibitem[Tayloe(2001)]{Tayloe:2001}
  Tayloe, D (2001), {\it Product detector and method therefor -- United States Patent No 6230000}
  \\\texttt{https://patentimages.storage.googleapis.com/ed/ec/5f/c214501bb441f1/US6230000.pdf}
\bibitem[Tayloe(2013)]{Tayloe:2013} 
  Tayloe, D (2013), {\it Ultra Low Noise, High Performance, Zero IF Quadrature Product Detector and Preamplifier},
  \\\texttt{https://wparc.us/presentations/SDR-2-19-2013/Tayloe\_mixer\_x3a.pdf}

\bibitem[Motorola(1996)]{Motorola:1996}
  Motorola (1996),
  {\it Analog Multiplexers/Demultiplexers High–Performance Silicon–Gate CMOS},
\\\texttt{http://www.om3bc.com/datasheets/74HC4051.PDF}
  
\end{thebibliography}

\end{document}

